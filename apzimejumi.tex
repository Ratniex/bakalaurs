\begin{description}
    \item[\ppt] Varbūtisks algoritms
    \item[$\m$]Ziņojumu telpa
    \item[$\k$]Atslēgu telpa
    \item[$Gen$]Atslēgu ģenerēšanas algoritms
    \item[$Enc$]Šifrēšanas algoritms
    \item[$Dec$]Atšifrēšanas algoritms
    \item[$Sig$]Parakstīšanas algoritms
    \item[$Ver$]Paraksta pārbaudīšanas algoritms
    \item[$P(A|B)$]Notikuma $A$ varbūtība, ja ir izpildījies notikums $B$
    \item[$x \leftarrow \a$]$x$ pieņem vērtību, kuru izdod varbūtiskais algoritms $\a$
    \item[$\oi^n$] Nuļļu un vieninieku virkņu telpa garumā $n$ 
    \item[$\oi^*$]Patvaļīga garuma nuļļu un vieninieku virkņu telpa
    \item[$\mathbf{1}^n$]Vieninieku virkne garumā $n$ (drošuma parametrs)
    \item[$U(X)$] Gadījuma lielums, kas raksturo vienmērīgu sadalījumu pār kopas $X$ elementiem
    \item[hash$(x)$] Jaucējfunkcijas vērtība pie $x$.
    \item[Blokķēde] Datu struktūra līdzīga sarakstam, kurā katrs elements satur iepriekšējā elementa hash vērtību.
    \item[Ķēde] Decentralīzēta publiski pieejama datu bāze, kas darbojas pēc saviem noteikumiem.
\end{description}
