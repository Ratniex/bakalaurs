Līdz ar datoru izveidi kriptogrāfija mainījās no praktiskas disciplīnas par teorētisku zinātni un tās īsajā pastāvēšanas vēsturē attīstība ir notikusi strauji un bez pārtraukumiem. Par blokķēdes pirmsākumiem uzskatāms 2008 gads, kad Satoši Nakamoto uzrakstīja Bitcoin tehnisko aprakstu, tātad tehnoloģija ir pastāvējusi nepilnus 9 gadus. Nopietna pāris simtu lapaspušu grāmata izdota 2015 gadā, kuru izdevās pāršķirstīt radīja iespaidu, ka ir novecojusi un slikti uzrakstīta. Pašlaik priekštatu par blokķēdēm vislabāk gūt no tehniskiem aprakstiem, ierakstiem blogos un informatīviem bukletiem. Augošais un relatīvi vecais kriptovalūtu tirgus liecina, par to, ka blokķēde ir nākotnes tehnoloģija, tomēr augstās maiņas kursa svārstības liecina, ka vēl nav zināms kura no ķēdes realizācijām kļūs par vispārpieņemto standartu. Praktiskas standartizācijas trūkums traucē arī matemātiskas formalizācijas izveidei.

Turpretī godīga apmaiņa ir vecāka un šaurāka problēma, kas ir daudzmaz izpētīta, padarot šo teoriju par labu atspēriena punktu, lai aprakstītu blokķēžu \textit{smart contracts} funkcionalitāti. Ņemot vērā plašo ticību blokķēžu vēstures nemaināmībai ir iespējams praktiski veikt godīgu apmaiņu jebkam, ko ir iespējams `uzlikt' uz ķēdes.

Ķēde ir populārs norēķinu līdzeklis noziedzinieku vidū pateicoties tās pseidonimitātei un necenzējamībai. Efektīvi vienīgais veids, kā aizliegt kriptovalūtas ir ar smagu interneta cenzūru.
