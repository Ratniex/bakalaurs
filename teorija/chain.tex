% Trūkst vienotas definīcijas par to, kas ir ķēde vispārīgā kontekstā. 
Tas ko sauc par ķēdi plašākā sabiedrībā tik stipri atšķirās no Bitcoin tehniskā apraksta\cite{nakamoto08} dotās definīcijas, ka tā vairs nav izmantojama. 
Lielākoties par ķēdi tiek saukta decentralizēta norēķinu sistēma, tomēr pa virsu šīm norēķinu sistēmām tiek veidoti arī citi servisi, kuriem ar naudas pārskaitījumiem nav nekāda sakara, piemēram domēna vārdu reģistra, sertefikātu autoritātes, vēlēšanu un citi servisi.\cite{namecoin} Arī šīs tiek sauktas par ķēdēm.
Turklāt esošās ķēžu implementācijas ir tik daudzveidīgas un ar atšķirīgu funkcionalitāti, ka ir grūti spriest kādās ir vispārīgas ķēdes īpašības.
Šajā nodaļā aplūkosim vairākas saistītas tēmas:
\begin{description}
    \item[Decentralizētas sistēmas]
        % torrenti, BGP, consensus nepieciešamība, anti spam nepieciešamība <- parakstu nepieciešamība
        \item[Digitālas valūtas]
        % double spending, te kļūst skaidrs, ka secība ir svarīga.
        % reusability
        \item[Blokķēdes realizācijas]
        % hash chain
        %% skip list
        %% merkle tree
        % consensus algorithms
        %% proof of work 
        %% proof of stake
        %% fba
        %% proof of ddos
\end{description}
\subsection{wow}
% trūkts vienotas blokķēdes definīcijas vispārīgā kontekstā, vēl jo vairāk trūkst pat vienotas izpratnes.
% ir vairākas eksistējošas ķēdes implementācijas un katrai no tām piemīt sava specifiska.
% ķēde ir vairāku tehnoloģiju apvienojums un tai piemīt vairākas svarīgas īpašības

% ir diezgan slikti ar blokķēdes definīcijām, jo risinājums, kuru tirgum piegādā bitcoin sastāv no vairākām komponentēm. valodas vulgarizācijas rezultātā orģināldarbā nakamoto par blokķēdi sauc kaut ko pilnīgi citu nekā ir pieņemts plašākā sabiedrībā. miskoncepcijas ir radījušas problēmas ne tikai biznesa cilvēkiem, bet arī tehniskajiem. ir mēģināts uzbūvēt vairākas perversas lietas un kods nav modulārs. hyperledger mēģina to vērst par labu uztaisot frame work
% uzbūvēt blokķēdes kā trešās puses matemātisko modeli ir grūti jo trūkst literatūras 
% Par \textbf{forumu} sauc tādu trešo pusi, kas uztur sarakstu ar saņemtajiem ziņojumiem to saņemšanas secībā un 
% aplūko sistēmu kur trešā puse nespēj paturēt datus slepenībā, bet nespēj uzmest un nav cenzēta. vienam no ziņojumiem arī jābūt transakcijai

% pameklēt fair exchange blockchain
% pameklēt blockchain abstraktu klasifikāciju iekš scholar
% ja neizdodas tad definēt: forumu, izdevniecību, tirgu, necenzētu

% viena no īpašībām ir spēja atrisināt Byzantine Generals Problem bet tas nav svarīgi šajā kontekstā
% kas ir DA0
