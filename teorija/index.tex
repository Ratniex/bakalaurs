Vispirms tiek uzskaitītas vispārzināmas kriptogrāfijas definīcijas. 
%Tām netiek doti izsmeļoši skaidrojumi, jo tiek pieņemts, ka lasītājs ar tām ir pazīstams.
Tiek padziļināti aplūkota godīgas apmaiņas problēma un tās praktiskā nozīme. Tiek aprakstīti fundamentālie kritēriji, kuri izpildās jebkurai digitālai valūtai. Beigās tiek aplūkotas galvenās idejas, uz kurām balstās ķēde. Tiek apskatīta vārda `blokķēde' izcelsme, un tiek salīdzināta vārda sākotnējā nozīme ar tā lietojumu plašākā sabiedrībā. Visbeidzot tiek uzdota ķēdes definīcija šī darba kontekstā.
