Šajā nodaļā tiek konspektēts minimālais kriptogrāfijas daudzums, lai darbs būtu uztverams, tomēr padziļinātas zināšanas ir ieteicamas un to iegūšana tiek atstāta lasītāja ziņā. Definīcijas un atsevišķi skaidrojumi tiek ņemti no\cite{pass10}.

Par \textbf{kriptosistēmu} sauc kopu $(\m, \k, Gen, Enc, Dec)$, kur
\begin{description}
    \item[$\m$]Ziņojumu telpa
    \item[$\k$]Atslēgu telpa
    \item[$Gen$]Atslēgu ģenerēšanas algoritms
    \item[$Enc$]Šifrēšanas algoritms
    \item[$Dec$]Atšifrēšanas algoritms
\end{description}

Par asimetrisku jeb publisko atslēgu kriptosistēmu sauc tādu, kurai atslēgu telpa sastāv no pāriem $(k_p, k_s)$, kur $k_p$ ir publiskā atslēga un $k_s$ ir slepenā atslēga. Šajās sistēmās šifrēšanas algoritmam ir atļauts izmantot publisko atslēgu, bet atšifrēšanas algoritmam - privāto. Turpmāk tiks runāts tikai par asimetriskām kriptosistēmām.

\textbf{Varbūtisks algoritms} jeb varbūtiska polinomiālā laikā ierobežota Tjūringa mašīna (Probabilistic polynomial-time Turing machine (\ppt)) ir Tjūringa mašīna, kurai papildus ievaddatu lentei ir dota lente, kura pēc katra soļa padod mašīnai jaunu gadījuma skaitli. Ātrdarbība tiek novērta no apakšas, pieņemot, ka jebkuriem ievaddatiem tiks uzģenerēti tādi gadījuma skaitļi, kas liks algoritmam darboties pēc iespējas ilgāk. Algoritmu sauc par \textbf{efektīvu}, ja eksistē tāds polinoms $P$, ka jebkuram ievaddatu izmēram $n$ tas apstājas ātrāk nekā $P(n)$ soļos. Varbūtiska algoritma rezultāts ir gadījuma lielums.

Varbūtisks algoritms $\a$ \textbf{izrēķina} funkciju $f$, ja
$$ \forall x : \quad P[\a(x) = f(x)] = 1 $$
Praksē tiek izmantoti arī algoritmi, kas retos gadījumos kļūdās, bet šī varbūtība ir tik maza, ka to var neņemt vērā.

Kriptosistēmu sauc par efektīvu, ja izpildās īpašības:
\begin{enumerate}
    \item $Gen(1^n)$ ir \ppt, kas izveido atslēgu $k$ jebkurai $n\in\n$ vērtībai.
    \item $Enc_k(m)$ ir \ppt, kas izveido šifrtektstu $c$ jebkurai atslēgai $k$ un ziņojumam $m$.
    \item $Dec_k(c)$ ir \ppt, kas dotam $k$ atšifrē ziņojumu $c$.
    \item Visiem ziņojumiem tiek atšifrēts tas pats, kas tika nošifrēts.
        $$ \forall n\in\n, m\in\m \quad P[k\leftarrow Gen(1^n): Dec_k(Enc_k(m)) = m] = 1 $$
\end{enumerate}

Par \textbf{pretinieku} sauc \ppt{}, kurš kā ievaddatus saņem kriptosistēmu, publisko atslēgu un drošuma parametru $1^n$.

Jebkurai kriptosistēmai prasa, lai no dotiem $m\in\m$ un $k\in\k$ būtu iespējams efektīvi ģenerēt šifrtekstu $c$, kā arī, lai no jebkura $c$ nebūtu iespējams noskaidrot $k$ vai $m$.

Par \textbf{vienvirziena funkciju} $f$ sauc tādu, kurai eksistē \ppt{} $\mathcal{C}$, kas aprēķina $f$, bet neeksistē \ppt{} $\a$, kuram 
$$ \forall y \in \text{Ran}(f): P[f(\a(y)) = y] = 1 $$
Vispārīga kriptogrāfijas teorija daudz balstās uz vienvirziena funkcijām, un tām ir smalkāks iedalījums, bet dotajai definīcijai ir maza teorētiskā vērtība.

Par jaucējfunkciju (hash function) sauc vienvirziena funkciju $h: \oi^* \to \oi^n$, kurai izpildās noturība pret sadursmēm. 
\begin{equation*}
    P[x, y \leftarrow \a : h(x) = h(y)] = 0
\end{equation*}
kur $\a$ ir pretinieks un $x\neq y$.

Par \textbf{parakstu sistēmu} sauc kortežu $(\m, \k, Gen, Sig, Ver)$, kuram izpildās
\begin{equation*}
    \forall m \in \m \;:\; P[k \leftarrow Gen: Ver_k(m, Sig_k(m))=\t] = 1
\end{equation*}
Šeit $\m$ ir ziņojumu telpa, $\k$ ir atslēgu telpa, $Gen$ ir atslēgu ģenerēšanas algoritms, $Sig$ ir parakstīšanas algoritms un $Ver$ ir paraksta pārbaudīšanas algoritms. Līdzīgi kā kriptosistēmu gadījumā $k$ sastāv no slepenās atslēgas $k_s$, kuru izmanto parakstītājs, un publiskās atslēgas $k_p$, kuru izmanto paraksta pārbaudei. Algoritma $Ver$ vērtību apgabals sastāv no divām vērtībām \textemdash{} derīgs $\t$ un nederīgs $\f$.

Par \textbf{parakstītu ziņojumu} sauc pāri $(m, s)$, kur $m\in \m$ ir ziņojums, bet $s$ ir paraksts. Atslēgu $k$ sauc par parakstītāju. Parakstītu ziņojumu $(m,s)$ sauc par derīgu, ja $Ver(m,s) =\t$.

Parakstu sistēmu sauc par drošu, ja
\begin{equation*}
    P[k \leftarrow Gen; (m,s) \leftarrow \a (k_p)\;:\; m\ni Q \wedge Ver_k(m,s) =\t] = 0
\end{equation*}
kur $\a$ ir pretinieks, kuram pieejams orākuls, kas, saņemot $m$, atbild ar $Sig_k(m)$, bet $Q$ ir saraksts ar jautājumiem, kuri tika uzdoti orākulam.\cite[p.~135]{pass10}
%paskaidrojums, kas ir orākuls

No kriptosistēmas $(\m', \k', Gen', Enc, Dec)$ un jaucējfunkcijas $h$ var uzkonstruēt parakstu sistēmu $(\m, \k, Gen, Sig, Ver)$
\begin{align*}
    & \m = \m' \\
    & \k = \left\{ (k_s, k_p) | (k_p, k_s) \in \k' \right\} \\
    & Gen = Swap(Gen') \\
    & Sig(m) = \big(m, Enc_k(h(m))\big)  \\
    & Ver(m,s) =
    \begin{cases}
        \t, &\text{ja}\; h(m) = Dec_k(s) \\
        \f, &\text{citādi}
    \end{cases}
\end{align*}
Ievēro, ka parakstu sistēmas privātā atslēga ir kriptosistēmas publiskā un otrādi, tāpēc no kriptosistēmas papildus nepieciešams nosacījums, ka no slepenās atslēgas un nošifrētiem ziņojumiem nevar izskaitļot publisko atslēgu.
