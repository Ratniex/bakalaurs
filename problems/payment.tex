% Kiberuzbrukums ar izspiedējvīrusu ir aktuāla mūsdienu pasaules problēma. 
Aplūkosim uzdevumu, kurā Alise ir hakeris un Bobs ir kritis par upuri Alises izspiedējvīrusam.
Bobam ir nošifrēti dati $Enc_k(d)$, bet Alisei ir atšifrēšanas atslēga $k_s$. Viņiem abiem ir zināmi šifrēšanas un atšifrēšanas algoritmi. Alise vēlas apmainīt atslēgu pret samaksu kriptovalūtā. Atslēga, kuru Bobs vēlas iegūt atbilst aprakstam, ja
\begin{equation*}
    Dec_k(Enc_k(d)) = d
\end{equation*}
Tomēr Bobam, kurš ir izspiedējvīrusa upuris, nav zināmi dati $d$. Labākajā gadījumā Bobam būs zināmi metadati par $d$, piemēram, dati veido pareizi formatētu partīciju un uz tās ir atrodami pdf faili, tad ir iespējams uzrakstīt funkciju $ValidMeta: \m \leftarrow \left\{ \t,\f \right\}$, kas pārbauda tamlīdzīgus nosacījumus. Tad teiksim, ka atslēga atbilst aprakstam, ja funkciju kompozīcija $ValidMeta \circ Dec_k$ atgriež patiesu vērtību.

Lai atvieglotu un standartizētu $ValidMeta$ funkciju Alises vīruss var pirms šifrēšanas datiem priekšā pierakstīt patvaļīgu konstantu simbolu virkni, tad $ValidMeta$ pārbauda, vai dati sākas ar šādu simbolu virkni. Savukārt dati $d$ sastāv no simbolu virknes un Boba biznesa loģikas datu apvienojuma.

Tehnisku iemeslu dēļ atšifrēt un pārbaudīt lielus datu apjomus uz ķēdes ir ilgi un dārgi, tāpēc nepieciešams šifrēt upura datus sekojošā veidā.
\begin{enumerate}
    \item Boba dati tiek sadalīti mazākos gabalos.
    \item Katram gabalam priekšā tiek pierakstīta konstanta simbolu virkne.
    \item Katrs gabals tiek nošifrēts.
    \item Šifrētie gabali tiek saglabāti uz diska tā, lai var atpazīt kur beidzas viens un sākas cits.
\end{enumerate}
Apmaiņa notiek sekojošā veidā:
\begin{enumerate}
    \item Bobs, ņemot vērā Alises vēlmes, izvēlas laiku $t_1$ pirms kura jānotiek apmaiņai.
    \item Bobs patvaļīgi izvēlas dažus no nošifrēto datu blokiem.
    \item Bobs uzkonstruē uz ķēdes kontu ar īpašībām:
        \begin{itemize}
            \item Ja tiek demonstrēta kriptogrāfiska atslēga $k$, kas pareizi atšifrē Boba izvēlētos blokus, tad valūta tiek pārskaitīta Alises kontam.
            \item Pēc $t_1$ valūta tiek pārskaitīta Boba kontam.
        \end{itemize}
    \item Alise demonstrē atšifrēšanas atslēgu.
\end{enumerate}

Praktisku realizāciju iespējams veikt uz Ethereum ķēdes, jo tā pieļauj patvaļīgu skaitļošanu. Konceptuālu kodu skatīt pielikumā.
% uzrakstīt kodu
