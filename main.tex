\documentclass[singlespacing, latexmargins]{ludis}

% % custom packages by Peteris Ratnieks
\usepackage{booktabs}
\usepackage{listings}
\lstset{numbers=left,
    stepnumber=1,
    firstnumber=1,
    numberfirstline=true
}
% \lstinputlisting[language=Python, firstline=2, lastline=22]{code.py}

% XeLaTeX atbalsts tiek pieslēgts ar šādām pakotnēm:
\usepackage{fontspec}
\usepackage{xunicode}
\usepackage{xltxtra} %

% Valodu atbalsts
\usepackage{polyglossia}
\setdefaultlanguage{latvian}
\setotherlanguages{english,russian}

% Fonti -- var rakstīt sistēmas fontu nosaukumus
% \setmainfont[Mapping=tex-text]{Times New Roman}
% \setsansfont[Mapping=tex-text]{Arial}
% \newfontfamily\russianfont{Times New Roman}

\usepackage{multirow}

\usepackage{hyperref}


% \def\r{\mathbb{R}}
% \def\c{\mathbb{C}}
% \def\k{\mathbb{K}}
% \def\z{\mathbb{Z}}
% \def\n{\mathbb{N}}
% \def\co{\mathsf{co}}

\def\m{\mathcal{M}}
\def\k{\mathcal{K}}
\def\a{\mathcal{A}}

\def\sumin{\sum_{i=1}^n}
\def\sumon{\sum_{i=0}^n}
\def\suminf{\sum_{i=1}^\infty}
\def\ppt{\textsf{PPT}}
% \def\p{\textsf{P}}

% \def\svp{\textsf{SVP} }
% \def\cvp{\textsf{CVP}}
% \def\svpg{\textsf{SVP$_\gamma$} }
% \def\cvpg{\textsf{CVP$_\gamma$} }
% \newcommand{\bra}[2]{\langle #1,#2 \rangle} 


\fakultate{Fizikas un Matemātikas}
\nodala{Matemātikas}
\nosaukums{Blokķēdes izmantošana kā trešā puse godīgas apmaiņas problēmā.}
\darbaveids{Bakalaura}
\DARBAVEIDS{BAKALAURA}
\autors{Pēteris Ratnieks}
\studapl{pr13005}
\vaditajs{Mg.dat. Kārlis Podiņš}
\recenzents{Dr.math. Raivis Bēts}
\vieta{Rīga}
\gads{2017}


\begin{document}

\maketitle

\begin{abstract-lv}
    not implemented

    \keywords{blokķēde, godīga apmaiņa, virtuālā nauda}
\end{abstract-lv}

\begin{abstract-en}
    not implemented

\keywords{blockchain, fair exchange, virtual currency}
\end{abstract-en}

\tableofcontents

\specnodala{Apzīmejumi}
\begin{description}
    \item[\ppt] Varbūtisks algoritms
    \item[$\m$]Ziņojumu telpa
    \item[$\k$]Atslēgu telpa
    \item[$Gen$]Atslēgu ģenerēšanas algoritms
    \item[$Enc$]Šifrēšanas algoritms
    \item[$Dec$]Atšifrēšanas algoritms
    \item[$P(A|B)$]Notikuma $A$ varbūtība, ja ir izpildījies notikums $B$
\end{description}


\specnodala{Ievads}
Tirdzniecība interneta vidē gūst arvien lielāku popularitāti. Ar to saistītas arī vairākas problēmas attiecībā pret godīgu apmaiņu starp divām savstarpēji neuzticīgām pusēm. Pieņemsim, ka Alise un Bobs abi vēlas parakstīt elektronisku dokumentu, bet katrs no viņiem otram neuzticās tapēc Alise nesūtīs Bobam savu parakstu pirms Bobs neatsūtīs viņai savējo un otrādi. Tādējādi viņi nekad neapmainīsies ar parakstiem, turklāt ir pierādīts, ka bez trešās puses iejaukšanās nav iespējams sasniegt stingri godīgu apmaiņu. \cite{pagnia99}
Ir piedāvāti dažādi problēmas risinājumi ar dažādām trešajām pusēm, bet ņemot vērā nesenos sasniegumus kriptovalūtu jomā uzskatu par nepieciešamu izpētīt risinājumus, kas trešās puses lomā novieto blokķēdi. 

Lai labāk saprastu godīgas apmaiņas problēmu norādīsim neformālu definīciju un aplūkosim vēl vienu piemēru.  
An exchange is fair if at the end of the exchange, either each player
receives the item it expects or neither player receives any additional
information about the other's item. \cite[p.~8]{asokan98}

Asokan, N. 1998. Fairness in electronic commerce . Ph. D. thesis, University of Waterloo.

Godīga apmaiņa ir tāda, ka pēc tās izpildes katrs dalībnieks iegūst ziņojumu, ko sagaidīja, vai arī neviens no dalībniekiem neguva papildus informāciju par citiem ziņojumiem.[Asokan 1998, p. 8]

Blokķēde (ķēde) ir datu struktūra līdzīga saistītam sarakstam, bet katrā posmā tiek ierakstīta iepriekšējā posma hash vērtība, tādējādi tajā nav iespējams ievietot jaunu posmu pa vidu, bet tos var pievienot tikai ķēdes galā. 
Tipiskākais ķēdes lietojums ir decentralizētās elektroniskās valūtās, lai novērstu iespēju melot par ķēdes vēsturi. 
Šī darba kontekstā ar ķēdi sapratīsim publisku datu bāzi, kurā ir iespējams rakstīt, bet nav iespējams mainīt vēsturi.
Pieņemsim, ka katrs ķēdes posms sastāv no vairākiem ierakstiem un katram ierakstam ir autors.
Lai izveidotu jaunu ierakstu jāizveido operācija uz kuras autors uzliek parakstu.

Ir pierādīts, ka atrisināt godīgas apmaiņas probēmu patvaļīgiem ziņojumiem ir neiespējami sistēmā kurā trūkst trešās puses. 
Problēmu var atvieglot pieļaujot trešās puses esamību un uzliekot ierobežojumus uz apmaināmajiem ziņojumiem.
Tiks aplūkots gadījums kur trešās puses lomā ir ķēdes.
Ziņojumi ir attiecīgo ķēžu operācijas ar derīgu parakstu.
Aprakstošais algoritms pārbauda paraksta derīgumu pret operāciju. 

Interneta forumos ir aprakstīts algoritms, kas pieļauj valūtu apmaiņu starp divām blokķēdēm.
Tā kā ir divas ķēdes ir arī divas operācijas un tiek panākts, lai izpildītos abas operācijas, vai arī neizpildītos neviena.
Varētu teikt, ka iesaistītās puses ir apmainījušās ar parakstītām ķēdes operācijām.
Šajā darbā tiks aplūkots kādas īpašības piemīt ķēdei, kā trešajai pusei, godīgas apmaiņas problēmā, kādi ierobežojumi piemīt operācijām kā arī formāli aprakstīts minētais algoritms un meklēts vispārīgāks teorētisks rezultāts, kas balstītos uz līdzīga principa.




\chapter{Teorijas apskats}
Vispirms tiek uzskaitītas vispārzināmas kriptogrāfijas definīcijas. 
%Tām netiek doti izsmeļoši skaidrojumi, jo tiek pieņemts, ka lasītājs ar tām ir pazīstams.
Tiek padziļināti aplūkota godīgas apmaiņas problēma un tās praktiskā nozīme. Tiek aprakstīti fundamentālie kritēriji, kuri izpildās jebkurai digitālai valūtai. Beigās tiek aplūkotas galvenās idejas, uz kurām balstās ķēde. Tiek apskatīta vārda `blokķēde' izcelsme, un tiek salīdzināta vārda sākotnējā nozīme ar tā lietojumu plašākā sabiedrībā. Visbeidzot tiek uzdota ķēdes definīcija šī darba kontekstā.


\section{Kriptogrāfijas definīcijas}
% ppt
% crypto system
% secure
Par \textbf{kriptosistēmu} sauc kopu $(\m, \k, Gen, Enc, Dec)$, kur
\begin{description}
    \item[$\m$]Ziņojumu telpa
    \item[$\k$]Atslēgu telpa
    \item[$Gen$]Atslēgu ģenerēšanas algoritms
    \item[$Enc$]Šifrēšanas algoritms
    \item[$Dec$]Atšifrēšanas algoritms
\end{description}

\textbf{Varbūtisks algoritms} (Probabilistic polynomial-time Turing machine \mbox{(\ppt))} ir Tjūringa mašīna kura spēj ģenerēt neatkarīgus gadījuma skaitļus. Ātrdarbība tiek novērta no apakšas pieņemot, ka tiks uzģenerēti tādi gadījuma skaitļi, kas liks algoritmam darboties pēc iespējas ilgāk. Algoritms atbilst definīcijai, ja eksistē tāds polinoms $P$, ka jebkuram ievaddatu izmēram $n$ tas apstājas ātrāk kā $P(n)$ soļos. Par algoritmiem, kuri atbilst definīcijai saka, ka tie ir \textbf{efektīvi}.

Varbūtisks algoritms $\a$ \textbf{aprēķina} funkciju $f$, ja 
$$ \forall x : \quad P[\a(x) = f(x)] = 1 $$

% \paragraph{Šanona drošuma definīcija.}

% Šifrtekstu neatšķiramības drošuma definīcija.




\section{Godīga apmaiņa}
Aplūkosim Alisi un Bobu. Pieņemsim, ka Alisei ir zināms ziņojums $m_a$, kas atbilst aprakstam $Desc_a$ kuru Bobs vēlas noskaidrot un Bobam ir ziņojums $m_b$, kuru vēlas Alise, turklāt viņi ir gatavi apmainīties ar ziņojumiem. Par \textbf{apmaiņu} sauc protokolu pēc kura izpildes abas puses iegūst vēlamos ziņojumus. 

Apmaiņu sauc par \textbf{godīgu}, ja pēc tās veikšanas visas puses ir ieguvušas vēlamos ziņojumus, vai arī neviena no pusēm nav ieguvusi papildus informāciju par ziņojumiem. Naiva apmaiņa ir sekojoša: Alise nosūta savu ziņojumu Bobam un tad Bobs nosūta savu ziņojumu Alisei. Šāda apmaiņa nav godīga, jo Bobs pēc ziņojuma saņemšanas var novirzīties no protokola un savu ziņojumu nesūtīt. Alisei un Bobam savā starpā sazinoties neizbēgami kāds no viņiem pirmais iegūs sev vēlamo ziņojumu un pārtrauks komunikāciju tapēc problēma nav atrisināma bez trešās puses iesaistīšanās apmaiņā.\cite{pagnia99}

Vienkāršības labad veiksim sekojošus pieņēmumus.
\begin{itemize}
    \item Visi komunikācijas kanāli ir droši. Tos nenoklausās.
    \item Visi komunikācijas kanāli ir uzticami. Katrs ziņojums garantēti nonāk pie saņēmēja īsā laikā.
    \item Alise un Bobs ir vienojušies par globālo laiku $t_1$ pēc kura jebkurš nenosūtīts ziņojums tiek uzskatīti par novirzīšanos no protokola.
    \item Trešā puse katram saņemtajam ziņojumam garantēti dod parakstītu apstiprinājumu, ka ziņojums tika saņemts, kur apstiprinājums sastāv no globālā laika un saņemtā ziņojuma.
\end{itemize}

% aprakstīt klasifikāciju fair exchange problēmām un risināt parakstus
Aplūkosim konkrētu piemēru, kur Alise un Bobs ir vienojušies par līgumu $l$, kuru abi vēlas elektroniski parakstīt un katram no viņiem vajag otra parakstu. Pieņemsim, ka Alisei ir atslēgu pāris $k_a$, tad
\begin{itemize}
    \item Ziņojums kuru vēlas Bobs: $m_a = s_a$, kur $(l, s_a) = Sig_{k_a}(l)$
    \item Apraksta funkcija $Desc_a(x) = Ver_{k_a}(l, x)$
\end{itemize}
Līdzīgi definē arī Boba aprakstu un ziņojumu.

Viens no veidiem kā atvieglot uzdevumu ir ieviest trešo pusi T caur kuru notiek komunikācija, savukārt Alise ir apmierināta ne tikai tad, ja saņem vēlamo ziņojumu, bet arī tad, ja spēj pierādīt, ka trešā puse ir pret viņu sazvērējusies.
Protokols, kas izpilda godīgu apmaiņu dotajā gadījumā ir šāds.
\begin{enumerate}
    \item A nosūta T ziņojumu $m_a$ un funkciju $Desc_b$.
    \item B nosūta T ziņojumu $m_b$ un funkciju $Desc_a$.
    \item Ja $Desc_a(m_a) = Desc_b(m_b) = \t$, tad T nosūta A un B atbilstošos ziņojumus.
\end{enumerate}
Ja A vai B nenosūta savu ziņojumu, tad T pēc laika $t_1$ paziņo abiem, ka maiņa tiek atcelta. Ja T nosūta B ziņojumu $m_a$, bet A nenosūta neko, tad A uzzinot, ka B ir ieguvis parakstītu līgumu spēs pierādīt, ka T ir sazvērējies pret viņu. Fakts, ka A vēl ir jānoskaidro, ka B ir ieguvis parakstītu līgumu ir mazliet negodīgi pret A. Dotais protokols pieprasa ļoti augstu uzticību T gan no A, gan B. Ir izveidoti daudz labāki protokoli, kas neļauj T uzzināt $m_a$ un $m_b$ vērtības, turklāt liek T iejaukties tikai nepieciešamības gadījumā, tomēr nav novērsts nelielais negodīgums attiecībā pret A.\cite{asokan98}

Atrastajos apmaiņas protokolos~\cite{asokan98,schunter00}% citation needed
trešā puse tiek aplūkota kā neuzticama, tomēr šī darba ietvaros tiks definēta būtiski atšķirīga trešā puse. Tā būs pilnībā uzticama, bet nespēj turēt saņemtos datus slepenībā, padarot iepriekš piedāvātos protokolus nederīgus.

% ir diezgan slikti ar blokķēdes definīcijām, jo risinājums, kuru tirgum piegādā bitcoin sastāv no vairākām komponentēm. valodas vulgarizācijas rezultātā orģināldarbā nakamoto par blokķēdi sauc kaut ko pilnīgi citu nekā ir pieņemts plašākā sabiedrībā. miskoncepcijas ir radījušas problēmas ne tikai biznesa cilvēkiem, bet arī tehniskajiem. ir mēģināts uzbūvēt vairākas perversas lietas un kods nav modulārs. hyperledger mēģina to vērst par labu uztaisot frame work
% uzbūvēt blokķēdes kā trešās puses matemātisko modeli ir grūti jo trūkst literatūras 
% Par \textbf{forumu} sauc tādu trešo pusi, kas uztur sarakstu ar saņemtajiem ziņojumiem to saņemšanas secībā un 
% aplūko sistēmu kur trešā puse nespēj paturēt datus slepenībā, bet nespēj uzmest un nav cenzēta. vienam no ziņojumiem arī jābūt transakcijai

% pameklēt fair exchange blockchain
% pameklēt blockchain abstraktu klasifikāciju iekš scholar
% ja neizdodas tad definēt: forumu, izdevniecību, tirgu, necenzētu


\section{Blokķēde}
% Trūkst vienotas definīcijas par to, kas ir ķēde vispārīgā kontekstā. 
Tas ko sauc par ķēdi plašākā sabiedrībā tik stipri atšķirās no Bitcoin tehniskā apraksta\cite{nakamoto08} dotās definīcijas, ka tā vairs nav izmantojama. 
Lielākoties par ķēdi tiek saukta decentralizēta norēķinu sistēma, tomēr pa virsu šīm norēķinu sistēmām tiek veidoti arī citi servisi, kuriem ar naudas pārskaitījumiem nav nekāda sakara, piemēram domēna vārdu reģistra, sertefikātu autoritātes, vēlēšanu un citi servisi.\cite{namecoin} Arī šīs tiek sauktas par ķēdēm.
Turklāt esošās ķēžu implementācijas ir tik daudzveidīgas un ar atšķirīgu funkcionalitāti, ka ir grūti spriest kādās ir vispārīgas ķēdes īpašības.
Šajā nodaļā aplūkosim ar ķēdēm saistītas tēmas \textemdash{} decentralizētas sistēmas, digitālas valūtas un Merkles koki. Beigās tiks aplūkotas konkrētas ķēdes realizācijas.

\subsection{Digitāla valūta}
% Ar digitālām valūtām ir saistītas sekojošas problēmas.
Naudas atkārtota iztērēšana (double spending) ir kritiskākā digitālu valūtu problēma, kas ir analoģiska fiziskas naudas viltošanai, bet tā kā digitāla valūta pēc savas būtības glabājas uz datora, tad to ir viegli nokopēt neskaitāmos eksemplāros. Vienīgais veids, kā novērst problēmu ir, ja pārdevējs nelaiž pircēju ārā no veikala kamēr nav saņēmis apstiprinājumu par naudas īpašnieka maiņu no vienota patiesības avota.\cite{frankel96}
Mūsdienās par patiesības avotu parasti kalpo kāda centrāla organizācija, piemēram Visa vai PayPal, tomēr vienots patiesības avots var būt arī jebkurš izkliedētas sistēmas dalībnieks, ja visi tās dalībnieki laika gaitā pieņems vienādu stāvokli.
Ar parakstītu ziņojumu palīdzību ir iespējams izveidot autorizētus maksājuma pieprasījumus un šeit nav būtiskas atšķirības starp centralizētu un decentralizētu risinājumu. Savukārt sistēmai saņemot divus pieprasījumus no kuriem katrs atsevišķi ir derīgs, bet abi kopā izpildīties nevar, ir jāpieņem lēmums par to kurš izpildās un kurš ne.
Centralizētās sistēmās hronoloģiski pirmais pieprasījums tiktu izpildīts, bet vēlākais tiktu atteikts. Diemžēl decentralizētā gadījumā nav objektīva secība kurā pienāk ziņojumi. Ir nepieciešams vienprātības (consensus) protokols, kas garantēs, ka visi maksājuma pieprasījumi tiek sakārtoti vienādā secībā visiem dalībniekiem, nodrošinot arī vienādu virsgrāmatas stāvokli.
Turklāt algoritmam jābūt noturīgam pret cenzūru un sabotāžu, lai tas būtu lietojams decentralizētā vidē.

\subsection{Izkliedētā skaitļošana (distributed computing)}
Aprakstīsim Bizantijas vienošanās problēmu (Byzantine agreement, Byzantine generals problem).
Pieņemsim, ka ir sistēma kas sastāv no tīklā saslēgtiem datoriem, kuri sazinās savā starpā sūtot ziņojumus. Mērķis ir vienoties par koordinētu tālāko rīcību, kas ir atkarīga no ārējiem apstākļiem, kuri iepriekš nav zināmi. Uzdevumu padara grūtu tas, ka starp datoriem var būt arī tādi, kas ļaunprātīgi cenšas sabotēt vienotu rīcību.
% Like non-federated Byzantine agreement, FBA addresses the problem of updating replicated
% state, such as a transaction ledger or certificate tree. By agreeing on what updates to
% apply, nodes avoid contradictory, irreconcilable states. We identify each update by a
% unique slot from which inter-update dependencies can be inferred. For instance, slots
% may be consecutively numbered positions in a sequentially applied log.

% Virsotne $v$ var pieņemt izmaiņas $x$ pozīcijā $i$, kad tā ir pieņēmusi izmaiņas 
% A node 𝑣 can safely apply update 𝑥 in slot 𝑖 when it has safely applied updates in all
% slots upon which 𝑖 depends and, additionally, it believes all correctly functioning nodes
% will eventually agree on 𝑥 for slot 𝑖. At this point, we say 𝑣 has externalized 𝑥 for slot 𝑖.
% The outside world may react to externalized values in irreversible ways, so a node
% cannot later change its mind about them.
Šāda decentralizēta sistēma risina Bizantijas vienošanās problēmu, tapēc katru sistēmas dalībnieku sauksim par \textbf{virsotni}.
Virsotni sauc par \textbf{godīgu}, ja tā ievēro vienprātības protokolu un veiksmīgi saņem un atbild uz ziņojumiem.
Virsotņu kopu sauc par \textbf{drošu}, ja katrām divām virsotnēm tās publicēs vienādas vērtības.
Virsotni sauc par \textbf{dzīvu}, ja spēj publicēt jaunas vērtības nepaļaujoties uz negodīgo virsotņu sadarbību.
Virsotņu kopu sauc par \textbf{pareizu}, ja tā ir droša un katra virsotne ir dzīva.
% ielikt no stellar whitepaper bildi 7lpp

% Protokolam jāpanāk, ka visas godīgās virsotnes laika gaitā publicēs tās pašas izmaiņas.

% Failu apmaiņa starp vienaudžiem (p2p), izmantojot BitTorrent protokolu, ir pierādījusi sevi kā praktiski necenzējamu sistēmu. Ja kāds fails tīklā kļūst populārs, tad ir pamats ticēt, ka tas nevar no turienes pazust. Veiksmīgākie mēģinājumi cīnīties pret dalīšanos BitTorrent tīklā ir saistīti ar uzbrukumiem tādiem serveriem, kas uztur sarakstu ar tīklā atrodamajiem failiem. Tomēr ir ieviesti strādājoši paplašinājumi BitTorrent protokolam, kas ļauj atrast failus decentralizētā vidē un iepriekšminētie serveri vairs nav nepieciešami.\cite{pouwelse08}


% \begin{description}
%     \item[Decentralizētas sistēmas] Torenti, Bizantijas ģenerāļu problēma, vienprātības nepieciešamība sakārtotas vēstures uzturēšanai.
%         % torrenti, BGP, consensus nepieciešamība, anti spam nepieciešamība <- parakstu nepieciešamība
%         \item[Digitālas valūtas]
%         % double spending, te kļūst skaidrs, ka secība ir svarīga.
%         % reusability
%         \item[Blokķēdes realizācijas]
%         % hash chain
%         %% skip list
%         %% merkle tree
%         % consensus algorithms
%         %% proof of work 
%         %% proof of stake
%         %% fba
%         %% proof of ddos
% \end{description}

% pameklēt fair exchange blockchain
% pameklēt blockchain abstraktu klasifikāciju iekš scholar
% ja neizdodas tad definēt: forumu, izdevniecību, tirgu, necenzētu

% viena no īpašībām ir spēja atrisināt Byzantine Generals Problem bet tas nav svarīgi šajā kontekstā
% kas ir DA0


\chapter{Godīgas apmaiņas uzdevumi}
Šajā nodaļā tiek aplūkotas praktiskas ar ķēdi saistītas godīgas apmaiņas problēmas.
Vispirms aplūkosim ACID starpķēžu transakcijas.
Pēc tam tiks iegūts vispārīgāks rezultāts, %ko tu gribi pateikt ar šo teikuma daļu?
un kā piemērs tiks izpētīta atšifrēšanas atslēgas publiskošana apmaiņā pret samaksu kriptovalūtā.

\section{Atomāras starpķēžu transakcijas}
% uzzīmēt apmaiņas shēmu
% vienai no ķēdēm ir nepieciešama spēja iesaldēt valūtu
% valūtu simboliskā nozīme
Pieņemsim, ka apmaiņu veic Alise un Bobs un vēl ir divas ķēdes C un D. Alisei ir kriptovalūta uz ķēdes C un Bobam ir kriptovalūta uz ķēdes D. Alise un Bobs vēlas apmainīties ar kriptovalūtām un viņi to var izdarīt publicējot ziņojumus uz ķēdēm. Apskatīsim protokolu, kas radies interneta forumos\cite{nolan13}, bet labāk aprakstīts un izmantots vēlākā darbā\cite{back14}. Lai izmantotu metodi nepieciešams nosacījums, ka abām ķēdēm ir aptuvens globālais laiks un transakcijas ir iespējams konstruēt tā, lai to derīgums būtu atkarīgs no ķēdes laika. Tas nepieciešams, lai līdzekļus uz ķēdes varētu iesaldēt.
\begin{enumerate}
    \item Alise izvēlas patvaļīgu skaitli $x$.
    \item Alise izveido uz ķēdes C jaunu kontu $T1$ un uzstāda tam sekojošus noteikumus.
        \begin{itemize}
            \item Ja ķēdes laiks ir pārsniedzis laiku, kas ir 48 stundas tālā nākotnē no šī brīža, tad visu valūtu no šī konta var pārskaitīt uz Alises kontu.
            \item Ja tiek demonstrēts tāds $x'$, ka $h(x') = h(x)$ tad valūtu no šī konta var pārskaitīt uz Boba kontu.
            \item Kontam pašam nav tiesību pārvietot savus līdzekļus.
        \end{itemize}
        Tādējādi Bobs var iegūt līdzekļus uz šī konta, ja viņš mazāk kā 48 stundu laikā noskaidro $x$.
    \item Alise pārskaita no sava konta uz $T1$ attiecīgo kriptovalūtas daudzumu.
    \item Bobs no $T1$ uzzina vērtību $h(x)$.
    \item Bobs izveido uz ķēdes D jaunu kontu $T2$ ar noteikumiem:
        \begin{itemize}
            \item Ja ķēdes laiks ir pārsniedzis laiku, kas ir 24 stundas tālā nākotnē no šī brīža, tad visu valūtu no šī konta var pārskaitīt uz Boba kontu.
            \item Ja tiek demonstrēts tāds $x'$, ka $h(x') = h(x)$ tad valūtu no šī konta var pārskaitīt uz Alises kontu.
            \item Kontam pašam nav tiesību pārvietot savus līdzekļus.
        \end{itemize}
    \item Bobs pārskaita uz $T2$ attiecīgo valūtas daudzumu.
    \item Alise uz ķēdes D demonstrē $x$ vērtību, tādējādi valūta no $T2$ tiek pārskaitīta uz Alises kontu ķēdē D.
    \item Bobs sekojot līdzi ķēdes D transakcijām noskaidro $x$.
    \item Bobs izmanto $x$, lai uz ķēdes C iegūtu valūtu no $T1$.
\end{enumerate}
Kad Alise demonstrē $x$ vērtību uz ķēdes Bobam ir vismaz 24 stundas laika, lai iegūtu valūtu uz ķēdes C.
Bobs var pārbaudīt D ķēdes stāvokli vēlāk, kā 24, bet ātrāk kā 48 stundas no apmaiņas sākuma, lai secinātu, vai Alise ir demonstrējusi $x$ un attiecīgi atsaldēt savus līdzekļus vai arī pabeigt apmaiņu.

Oriģināldarbā apmaiņas plūsma ir mulsinošāka, jo tā ir rakstīta priekš Bitcoin \textit{smart contract} valodas. Pielikumā ir kods priekš attiecīgajām transakcijām, kur C ķēdes vietā ir Stellar, bet D ķēdes vietā ir Ethereum.
% uzrakstīt kodu
Metode ir universālāka nekā sākumā sķiet, jo ķēdes valūtai var būt arī simboliska nozīme, tā var apzīmēt īpašumtiesības fiziskās pasaules lietām, uzņēmumu daļas un tamlīdzīgi.\cite{rosenfeld12}
Apmaiņu var veikt arī vienas ķēdes ietvaros un tas ir noderīgi, ja ķēde neatļauj mainīt valūtu pašu pret sevi, kā atomāru operāciju.


\section{Samaksa par atšifrēšanas atslēgām}
% Kiberuzbrukums ar izspiedējvīrusu ir aktuāla mūsdienu pasaules problēma. 
Aplūkosim uzdevumu, kurā Alise ir hakeris un Bobs ir kritis par upuri Alises izspiedējvīrusam.
Bobam ir nošifrēti dati $Enc_k(d)$, bet Alisei ir atšifrēšanas atslēga $k_s$. Viņiem abiem ir zināmi šifrēšanas un atšifrēšanas algoritmi. Alise vēlas apmainīt atslēgu pret samaksu kriptovalūtā. Atslēga, kuru Bobs vēlas iegūt atbilst aprakstam, ja
\begin{equation*}
    Dec_k(Enc_k(d)) = d
\end{equation*}
Tomēr Bobam, kurš ir izspiedējvīrusa upuris, nav zināmi dati $d$. Labākajā gadījumā Bobam būs zināmi metadati par $d$, piemēram, dati veido pareizi formatētu partīciju un uz tās ir atrodami pdf faili, tad ir iespējams uzrakstīt funkciju $ValidMeta: \m \leftarrow \left\{ \t,\f \right\}$, kas pārbauda tamlīdzīgus nosacījumus. Tad teiksim, ka atslēga atbilst aprakstam, ja funkciju kompozīcija $ValidMeta \circ Dec_k$ atgriež patiesu vērtību.

Lai atvieglotu un standartizētu $ValidMeta$ funkciju Alises vīruss var pirms šifrēšanas datiem priekšā pierakstīt patvaļīgu konstantu simbolu virkni, tad $ValidMeta$ pārbauda, vai dati sākas ar šādu simbolu virkni. Savukārt dati $d$ sastāv no simbolu virknes un Boba biznesa loģikas datu apvienojuma.

Tehnisku iemeslu dēļ atšifrēt un pārbaudīt lielus datu apjomus uz ķēdes ir ilgi un dārgi, tāpēc nepieciešams šifrēt upura datus sekojošā veidā.
\begin{enumerate}
    \item Boba dati tiek sadalīti mazākos gabalos.
    \item Katram gabalam priekšā tiek pierakstīta konstanta simbolu virkne.
    \item Katrs gabals tiek nošifrēts.
    \item Šifrētie gabali tiek saglabāti uz diska tā, lai var atpazīt kur beidzas viens un sākas cits.
\end{enumerate}
Apmaiņa notiek sekojošā veidā:
\begin{enumerate}
    \item Bobs, ņemot vērā Alises vēlmes, izvēlas laiku $t_1$ pirms kura jānotiek apmaiņai.
    \item Bobs patvaļīgi izvēlas dažus no nošifrēto datu blokiem.
    \item Bobs uzkonstruē uz ķēdes kontu ar īpašībām:
        \begin{itemize}
            \item Ja tiek demonstrēta kriptogrāfiska atslēga $k$, kas pareizi atšifrē Boba izvēlētos blokus, tad valūta tiek pārskaitīta Alises kontam.
            \item Pēc $t_1$ valūta tiek pārskaitīta Boba kontam.
        \end{itemize}
    \item Alise demonstrē atšifrēšanas atslēgu.
\end{enumerate}

Praktisku realizāciju iespējams veikt uz Ethereum ķēdes, jo tā pieļauj patvaļīgu skaitļošanu. Konceptuālu kodu skatīt pielikumā.
% uzrakstīt kodu


\specnodala{Rezultāti}
Tika risināta praktiska problēma kurā uzbrucējs ir nošifrējis upura cieto disku un vēlas apmainīt atšifrēšanas atslēgu pret izpirkuma maksu kriptovalūtā. Tika uzrakstīts kods, kas konceptuālā līmenī demonstrē apmaiņas iespējamību.


\specnodala{Secinājumi}
Līdz ar datoru izveidi kriptogrāfija mainījās no praktiskas disciplīnas par teorētisku zinātni, 
% "no praktiskas disciplīnas par teorētisku zinātni" pieņemu, ka arī bez datoriem cilvēki domāja par kriptogrāfiju matemātikā un arī ar datoriem kriptogrāfija tiek darīta praktiski. ~ sapratu, ko domāji, bet ir ambiguity un var interpretēt visādi. izlabo.
un tās īsajā pastāvēšanas vēsturē attīstība ir notikusi strauji un bez pārtraukumiem. Par blokķēdes pirmsākumiem uzskatāms 2008. gads, kad Satoši Nakamoto uzrakstīja Bitcoin tehnisko aprakstu, tātad tehnoloģija ir pastāvējusi nepilnus 9 gadus. Nopietna pāris simtu lapaspušu grāmata izdota 2015. gadā, kuru izdevās pāršķirstīt 
%	"izdevās pāršķirstīt"? mēģini atrast tādus vārdus, lai lasītājam nešķiet ka tu darīji literatūras analīzi ironiski
radīja iespaidu, ka ir novecojusi un slikti uzrakstīta.
%	"slikti uzrakstīta" ir tavs viedoklis. objektīvi pamato, kāpēc slikti uzrakstīta, kas tai kaiš, kādi trūkumi.

Pašlaik priekštatu par blokķēdēm vislabāk gūt no tehniskiem aprakstiem, ierakstiem blogos un informatīviem bukletiem. Augošais un relatīvi vecais kriptovalūtu tirgus liecina, par to, ka blokķēde ir nākotnes tehnoloģija, tomēr augstās maiņas kursa svārstības liecina, ka vēl nav zināms, kura no ķēdes realizācijām kļūs par vispārpieņemto standartu. Praktiskas standartizācijas trūkums traucē arī matemātiskas formalizācijas izveidei.

Turpretī godīga apmaiņa ir vecāka un šaurāka problēma, kas ir daudzmaz 
% ieliec "daudzmaz" vietā kādu ne tik casual vārdu, kas viennozīmīgi/sakarīgāk apraksta izpētīšanas līmeni
izpētīta, padarot šo teoriju par labu atspēriena punktu, lai aprakstītu blokķēžu \textit{smart contracts} funkcionalitāti. Ņemot vērā plašo ticību blokķēžu vēstures nemaināmībai ir iespējams praktiski veikt godīgu apmaiņu jebkam, ko ir iespējams `uzlikt' uz ķēdes.

Ķēde ir populārs norēķinu līdzeklis noziedzinieku vidū pateicoties tās pseidonimitātei un necenzējamībai. Efektīvi vienīgais veids kā aizliegt kriptovalūtas ir smaga interneta cenzūra.


\specnodala{Pateicības}
\begin{itemize}
    \item Viktorijai Leimanei par valodas salabošanu.
    \item GNU projektam par kvalitatīvajiem rīkiem, kas palīdzēja darba tapšanā.
    \item Jānim Valeinim par \LaTeX{} stila izstrādi.
\end{itemize}


\literatura{main}

\appendix
\chapter{Izveidoto programmu kods}
\input{pielikumi/kods1.tex}
% \input{pielikumi/kods2.tex}

\end{document}
